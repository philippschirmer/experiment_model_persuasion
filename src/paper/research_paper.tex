\documentclass[11pt, a4paper, leqno]{article}
\usepackage{a4wide}
\usepackage[T1]{fontenc}
\usepackage[utf8]{inputenc}
\usepackage{float, afterpage, rotating, graphicx}
\usepackage{epstopdf}
\usepackage{longtable, booktabs, tabularx}
\usepackage{fancyvrb, moreverb, relsize}
\usepackage{eurosym, calc}
% \usepackage{chngcntr}
\usepackage{amsmath, amssymb, amsfonts, amsthm, bm}
\usepackage{caption}
\usepackage{mdwlist}
\usepackage{xfrac}
\usepackage{setspace}
\usepackage{xcolor}
\usepackage{subcaption}
\usepackage{minibox}
% \usepackage{pdf14} % Enable for Manuscriptcentral -- can't handle pdf 1.5
% \usepackage{endfloat} % Enable to move tables / figures to the end. Useful for some submissions.



\usepackage{natbib}
\bibliographystyle{rusnat}




\usepackage[unicode=true]{hyperref}
\hypersetup{
    colorlinks=true,
    linkcolor=black,
    anchorcolor=black,
    citecolor=black,
    filecolor=black,
    menucolor=black,
    runcolor=black,
    urlcolor=black
}


\widowpenalty=10000
\clubpenalty=10000

\setlength{\parskip}{1ex}
\setlength{\parindent}{0ex}
\setstretch{1.5}


\begin{document}

\title{Experiment Model Persuasion\thanks{Carolina Alvarez Garavito, Philipp Schirmer, University of Bonn. Email: \href{mailto:schirmerphilipp@gmail.com}{\nolinkurl{schirmerphilipp [at] gmail [dot] com}}.}}

\author{Carolina Alvarez Garavito, Philipp Schirmer}

\date{
    {\bf Preliminary -- please do not quote}
    \\[1ex]
    \today
}

\maketitle


\begin{abstract}
    VERY Preliminary \\

    We develop and implement a novel experimental design to test a theory of model persuasion using the 
    experimental platform oTree. To achieve this, we make use Python and JavaScript code as well as 
    employing html and CSS to make the experiment visually appealing.

    In a game between a sender and possibly multiple receivers, the sender can access an interactive toolbox of
    different model choices that can be used to persuade a receiver. The model messages contain no
    new information per se, and should hence not move the receiver's belief. In a setting that mimics real-life
    financial chart analysis, we can test several hypotheses concerned with "sense-making" persuasion by
    a potentially biased sender.
\end{abstract}
\clearpage

\section{Introduction} % (fold)
\label{sec:introduction}

If you are using this template, please cite this item from the references: \cite{GaudeckerEconProjectTemplates}

\cite{Schelling69} example in the code is taken from \cite{StachurskiSargent13}

Test.

The decision rule of an agent is the following:
\begin{align*}
    \text{move} & \quad \text{if} \quad n_\text{neighbours} < 4 \\
    \text{stay} & \quad \text{if} \quad n_\text{neighbours} \geq 4
\end{align*}


\begin{figure}
    \caption{Segregation by cycle in the baseline \cite{Schelling69} model as in the \cite{StachurskiSargent13} example
    }

    \includegraphics[width=\textwidth]{../../bld/figures/schelling_baseline}

\end{figure}


\begin{figure}
    \caption{Segregation by cycle in the baseline %\citet{Schelling69} model, limiting the number of potential moves per period to two
    }

    \includegraphics[width=\textwidth]{../../bld/figures/schelling_max_moves_2}

\end{figure}

% section introduction (end)





\bibliography{refs}



% \appendix

% The chngctr package is needed for the following lines.
% \counterwithin{table}{section}
% \counterwithin{figure}{section}

\end{document}
